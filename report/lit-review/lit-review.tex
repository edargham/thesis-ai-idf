\section{Literature Review}

In the paper, ``Machine Learning for Projecting Extreme Precipitation Intensity for Short Durations in a Changing Climate''~\cite{geosciences9050209}, Hu H. and Ayyub Bilal M. propose a machine learning solution to temporaly downscale daily rainfall data and leverage the downscaled data to generate IDF curves. The data used for this study was sourced from daily rainfall data obseved from two thousand stations across the United States of America. The authors highlighted that most of the downscaling on rainfall data took place in the spacial domain, thus they propose to do the same on the temporal domain.~\cite{geosciences9050209}

The primary input to the machine learning model is the daily rainfall data, as the authors stated that many global climate models (GCMs) and regional climate models (RCMs) offer only daily rainfall data rather than sub-daily precipitation data. This apporach makes the method widely applicable to leverage readily available daily climate projections to generate IDF curves.~\cite{geosciences9050209} using the daily rainfall data as input, and the observed subdaily resolutions as labels, the ML models would learn to estimate the downscaled precipitation data, which then could be used to construct IDF curves for different durations using the known distributions.

The authors in this paper used machine learning in the preprocessing of the IDF generation process. The data is not being used to replace any of the statistical distribution used to generate IDF curves from  annual maximas and extreme rainfall events, but rather in the steps before that to prepare the data. Furthermore, the authors relied on data from two-thousand ground stations in the United States, and did not leverage any remotely sensed precipitation data for this task.

\vspace{1em}

In the paper ``Utilizing Machine Learning and Deep Learning for Precise Intensity-Duration-Frequency (IDF) Curve Predictions''~\cite{idfkoya} Ameen, Sheeraz M. \emph{et.~al.} compared several machine learning (ML) and deep learning (DL) models with traditional statistical models using the Gumbel distribution to construct IDF curves.

The initial intuition is that Gumbel only performs well on stationary data that doesn't change over time, but fails to capture the changes in rainfall trends due to climate change, something that ML and DL are well suited for.~\cite{idfkoya}

The authors collected rainfall data in Koya, Iraq from 2005 to 2022, from multiple governmental and institutional sources, consisting of 5-minute, 10-minute, 20-minute, 30-minute, 60-minute, 120-minute, 180-minute, 360-minute, 720-minute and 1440-minute intervals. Using this data, the authors from one end, used Gumbel to construct the IDF curves to establish a benchmark, and from the other end, train Linear Regression (LR), Support Vector Regression (SVR) and Long Short-Term Memory (LSTM) models to predict the IDF curves.~\cite{idfkoya}

To train the models, the authors split the rainfall data into a training set, using the data from 2005 to 2015 and a validation set from 2016 to 2022. The metrics used to evaluate each model were the Mean Absolute Error (MAE), Root Mean Square Error (RMSE) and R-squared (R$^{2}$) values. The study revealed that the LSTM model outperformed the other ML and DL model as well as the Gumbel distribution, achieving the lowest RMSE (1.44 mm/hr), lowest MAE (0.81 mm/hr) and the Highest R$^{2}$ (0.99) values.~\cite{idfkoya}

The authors in this study however, relied on governmental and station data sources, and did not source any data using remote sensing technologies or satellite data.

\vspace{1em}

