\subsubsection{Artificial Neural Networks (ANN)}

The ANN model is trained to predict the log-transformed annual maximum rainfall intensity using both the log-transformed event duration input. The dataset is scaled with min-max scaling and split into 50\% for training and 50\% for validation to ensure robust generalization. The neural network architecture is provided in Table~\ref{tab:hyperparameters}.

\vspace{1em}

Model training is performed using the AdamW optimizer with a learning rate of 0.002 over 3000 epochs, minimizing mean squared error loss. After training, predictions are inverse-transformed and exponentiated to recover the original intensity scale. Model performance is evaluated using RMSE, MAE, NSE, and R$^{2}$ metrics across all return periods, and the resulting ANN-based IDF curves are compared visually and quantitatively against the Gumbel-distribution-based benchmarks.

\vspace{1em}

The UAT states that a feedforward neural network with at least one hidden layer and a sufficient number of neurons can approximate any continuous function on a compact domain. As a result, ANNs are particularly well-suited for predicting IDF curves, even when the underlying hydrological processes are highly nonlinear and difficult to model analytically.