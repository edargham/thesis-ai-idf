\subsubsection{Support Vector Machines (SVM)}
The SVM model was trained to predict the log-transformed annual maximum rainfall intensity from the log-transformed event duration, using a Support Vector Regression (SVR) approach. The input data was standardized, and a grid search with 5-fold cross-validation was employed to optimize the SVR hyperparameters. The best-performing model was then fitted to the training set, and its predictions were inverse-transformed and exponentiated to recover the original intensity scale.

\vspace{1em}

Model performance was evaluated on the test set using root mean squared error (RMSE), mean absolute error (MAE), and the Nash-Sutcliffe Efficiency (NSE) coefficient. IDF curves were generated for various return periods by scaling the base SVM predictions with empirically derived frequency factors, and compared against Gumbel-distribution-based IDF curves for validation.

\vspace{1em}

The selected hyperparameters for the SVM model are as follows:

\begin{align*}
\text{Kernel} &: \text{RBF} \\
C &: 10.0 \\
\epsilon &: 0.01 \\
\gamma &: 1.0 \\
\end{align*}

The SVM model's ability to capture the non-linear relationship between rainfall intensity and event duration was assessed, demonstrating its effectiveness in modeling complex hydrological processes. The model's predictions were visualized against the Gumbel-based IDF curves, showing a good fit for most return periods.