\subsection{Construction and Training}
The IDF curves are constructed using the Gumbel Distribution which was fitted to the yearly maximum intensities using the maximum likelihood estimation (MLE) method for each duration to detrmine the location and scale parameters.
Using these parameters, The return period intensities were calculated through the Gumbel inverse cumlative distribution function, resulting in IDF curves that serve as the statistical benchmark for evaluating the ML models.

\vspace{1em}

To train the ML and DL models, we take as input the duration of the event, and as output the maximum intensity of the duration for a year in the dataset. The year is dropped from the input as it prevents the model from properly learning the duration-intensity patterns~\cite{14}. For TCNs and TCANs, the data rearranged into proper sequences as input. Additionally, 60\% of the data is allocated for fitting the models to the data and 40\% for validating the models to ensure the models properly generalize. 

\vspace{1em}

For each ML and DL model, the detailed hyperparameter configuration used for training will be provided in the following sections, as well as the optimization strategies and validation procedures to ensure reproducibility and robust model performance.

\subsubsection{Support Vector Machines (SVM)}
The SVM model is trained to predict the log-transformed annual maximum rainfall intensity from the log-transformed event duration, using a Support Vector Regression (SVR) approach. The input data is scaled using min-max scaling, and a grid search with 5-fold cross-validation was employed to optimize the SVR hyperparameters. The best-performing model is then fitted to the training set, and its predictions were inverse-transformed and exponentiated to recover the original intensity scale.

\vspace{1em}

Model performance is evaluated on the test set using root mean squared error (RMSE), mean absolute error (MAE), Coefficient of Determination (R$^{2}$), and the Nash-Sutcliffe Efficiency (NSE) coefficient. IDF curves are generated for various return periods by scaling the base SVM predictions with empirically derived frequency factors, and compared against Gumbel-distribution-based IDF curves for validation.

\vspace{1em}

The SVM model's ability to capture the non-linear relationship between rainfall intensity and event duration is assessed, demonstrating its effectiveness in modeling complex hydrological processes. The model's predictions are visualized against the Gumbel-based IDF curves, showing a good fit for most return periods.
\subsubsection{Artificial Neural Networks (ANN)}

The ANN model was trained to predict the log-transformed annual maximum rainfall intensity using both the log-transformed event duration and log-transformed return period as inputs. The dataset was standardized and split into 50\% for training and 50\% for validation to ensure robust generalization. The neural network architecture consisted of two hidden layers with 24 neurons each, ReLU activations, and dropout regularization to prevent overfitting.

\vspace{1em}

Model training was performed using the Adam optimizer with a learning rate of 0.0001 over 2500 epochs, minimizing mean squared error loss. After training, predictions were inverse-transformed and exponentiated to recover the original intensity scale. Model performance was evaluated using RMSE, MAE, and R² metrics across all return periods, and the resulting ANN-based IDF curves were compared visually and quantitatively against the Gumbel-distribution-based benchmarks.

\vspace{1em}

The UAT states that a feedforward neural network with at least one hidden layer and a sufficient number of neurons can approximate any continuous function on a compact domain. As a result, ANNs are particularly well-suited for predicting IDF curves, even when the underlying hydrological processes are highly nonlinear and difficult to model analytically.
\subsubsection{Temporal Convolutional Networks (TCN) and Variants with Sparse Attention (TCAN)}

The TCN model is implemented to predict the log-transformed annual maximum rainfall intensity using sequences of duration as input features. Each input is formatted as a sequence of length 5, with features scaled with min-max scaling for stable training. The model architecture consists of two dilated 1D convolutional layers with residual connections, followed by global average pooling and a linear output layer, resulting in a highly parameter-efficient design.

\vspace{1em}

The model is trained using the AdamW optimizer with a learning rate of 0.002 and weight decay of 5e-5, for up to 2000 epochs with early stopping based on validation loss. A hybrid loss consisting of weighted mean squared error and mean absolute error is used to allow the model to appropriatly penalize outliers without completely ruling them out, and gradient clipping was applied for stability. Model performance is assessed using RMSE, MAE, NSE, and R$^{2}$ metrics, and the generated IDF curves are compared against the Gumbel-distribution-based benchmarks for validation.

\vspace{1em}

The TCN leverages dilated convolutions and residual connections to efficiently capture both local and long-range dependencies in the duration-intensity relationship, which is essential for modeling the complex temporal structure of rainfall events. Its lightweight architecture enables high accuracy with minimal parameters, making it suitable for hydrological modeling tasks with limited data.

\vspace{1em}

The TCAN model extends the TCN\@ by incorporating a lightweight multi-head self-attention mechanism after the dilated convolutional layers. This addition allows the network to better capture long-range dependencies and interactions between different temporal positions in the input sequence. The input preparation and scaling procedures mirror those used for the TCN\@, with the adjustments to hyperparameters detailed in Table~\ref{tab:hyperparameters}.

\vspace{1em}

Training of the TCAN follows the same protocol as the TCN, utilizing the AdamW optimizer, early stopping, but however, uses MSE loss to ensure robust generalization. Model evaluation is conducted using the same suite of metrics (RMSE, MAE, NSE, R$^{2}$), and the predicted IDF curves are benchmarked against the Gumbel-based results.

\vspace{1em}

The integration of sparse attention enables the TCAN to focus on the most relevant features across the input sequence, enhancing its ability to model complex, nonlinear relationships in rainfall data. This makes the TCAN particularly effective for generating accurate IDF curves, especially in scenarios where capturing subtle temporal dependencies is critical.
