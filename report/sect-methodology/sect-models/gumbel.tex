\subsection{Gumbel Distribution}
For the baseline, the Gumbel distribution is used to construct the IDF curves for each duration and return period.
The Gumbel distribution, also known as the extreme value type I distribution, is widely used in hydrological applications for modeling annual maximum precipitation values and constructing intensity-duration-frequency (IDF) curves and is particularly suitable for extreme precipitation analysis as it represents the limiting distribution of the maximum of a large number of independent and identically distributed exponential-type random variables~\cite{}.
The Gumbel distribution is defined by its cumulative distribution function (CDF) and probability density function (PDF):

\begin{equation}
F(x) = e^{-e^{-(x - \mu)/\beta}}
\end{equation}

\begin{equation}
f(x) = \frac{1}{\beta} e^{-(x - \mu)/\beta} e^{-e^{-(x - \mu)/\beta}}
\end{equation}

where:
\begin{itemize}
  \item $x$ is the precipitation intensity value.
  \item $\mu$ is the location parameter.
  \item $\beta$ is the scale parameter.
\end{itemize}
The parameters $\mu$ and $\beta$ are estimated from the maximum precipitation value for each year and duration using the method of moments or maximum likelihood estimation (MLE).

Once the parameters are estimated, the Gumbel distribution can be used to derive the IDF curves for each duration and return period.