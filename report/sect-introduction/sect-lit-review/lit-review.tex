Machine learning (ML) and deep learning (DL) methods have only recently started to appear in literature as an alternative to the traditional statistical methods used to construct IDF curves~\cite{idfkoya}, this trend still needs to gain traction, and more studies are needed to explore the potential of these methods. Per the universal approximation theorem (UAT), deep learning models are by design capable of capturing non-linear relationships in data, and are capable to estimate any function or distribution given enough data~\cite{Goodfellow2016-sect6.4.1}. However, artificial neural networks on their own still struggle with modeling stochastic processes and uncertainty inherent in some datasets~\cite{thacker2020fundamentalissuesregardinguncertainties}. This has led to the development of neural network architectures that are specifically designed to handle stochastic processes and uncertainty, some by design, such as Bayesian Neural Networks (BNN) and Temporal Convolutional Attention Networks (TCANs).~\cite{lin2021tcan, goan2020} 

\vspace{1em}

ML techniques were initially applied to precipitation data preprocessing and temporal disaggregation tasks rather than direct IDF curve construction. Artificial neural networks first emerged in hydrology applications during the early 1990s, primarily focusing on rainfall-runoff modeling, stream flow forecasting, and water quality analysis~\cite{doi:10.1061}. Early applications of neural networks to precipitation disaggregation began appearing in the literature around 2000, with studies exploring the use of feed-forward neural networks and competitive learning algorithms to disaggregate hourly rainfall data into sub-hourly time increments~\cite{doi:10.1061}. Using machine learning and deep learning methods to construct IDF curves is a relatively new approach to achieve adaptive IDF curves, with only a handful of studies that have emerged recently exploring this alternative~\cite{idfkoya}.

\vspace{1em}

Using the daily rainfall data as input, and the observed subdaily resolutions as labels, the ML models would learn to estimate the downscaled precipitation data, which then could be used to construct IDF curves for different durations using the known distributions~\cite{geosciences9050209}. This was achieved by machine learning in the preprocessing of the IDF generation process while relying on data from multiple ground stations, but did no remotely sensed precipitation data was leveraged for this task. Compared to traditional statistical methods, machine learning models have shown promising result in being able to construt IDF curves. Deep learning models, especially Long-Short Term Memory (LSTM) Recurrent Neural Networks (RNN), were shown to be a more accurate and reliable approach for predicting rainfall patterns and IDF curves than conventional methods~\cite{idfkoya}.