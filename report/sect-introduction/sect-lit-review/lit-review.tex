The construction of accurate Intensity-Duration-Frequency (IDF) curves necessitates high-resolution temporal precipitation data, particularly at sub-hourly intervals. Such data is critical for the design of urban drainage systems and for robust flood risk assessments, as short-duration extreme rainfall events can precipitate unforeseen natural disasters, potentially resulting in significant loss of life and considerable economic costs~\cite{hess-28-375-2024}.
In many regions, particularly within developing countries, the limited density of rain-gauge networks constrains spatial coverage and consequently compromises the accuracy of rainfall estimates~\cite{basumatary2016}. This scarcity in high-resolution data poses considerable challenges for engineers, as it hinders the acquisition of sub-hourly precipitation measurements required for the reliable construction of IDF curves, often necessitating the use of suboptimal IDF relationships~\cite{basumatary2016}.

The continuous advancement in satellite precipitation retrieval makes it increasingly important to develop methods that incorporate these datasets in constructing IDF curves, especially in regions where in-situ rainfall observations are sparse and have insufficient record length as it offers a credible and often superior alternative source for sub-hourly precipitation information in data-sparse regions~\cite{ombadi2018}. Satellite-based precipitation products such as the Global Precipitation Measurement (GPM) Integrated Multi-satellitE Retrievals for GPM (IMERG), the Global Satellite Mapping of Precipitation Moving Vector with Kalman Filter (GSMaP-MVK), the Precipitation Estimation from Remotely Sensed Information using Artificial Neural Networks (PERSIANN), and others have demonstrated effectiveness in estimating IDF curves, with some products showing strong correlation with ground observations~\cite{ombadi2018, rs14195032}. Half-hourly precipitation data from GPM-IMERG, for instance, was used to derive regional intensity–duration–frequency (IDF) curves with reasonable accuracy—especially when using the Gumbel distribution.~\cite{rs14195032}

\vspace{1em}

The statistical distribution functions used to construct IDF curves have been the subject of extensive research, and were found to be very effective in designing infrastructure under a stationary climate assumptions. However, the elements of variability and uncertainty in precipitation patterns have become a cause of concern, especially with the increasing impact of climate change on precipitation patterns~\cite{Cheng2014}. This calls into question the readiness of the current urban infrastructure to deal with these changes, especially since the stationarity climate assumption may lead to an underestimation of extreme precipitation by as much as 60\%~\cite{Cheng2014}. This has led to a growing interest to develop IDF construction methods that no longer rely on the stationarity assumption, but instead take into consideration the changing climate and precipitation patterns using methods such as bayesian inference or other distributions entirely~\cite{Cheng2014, hess-2020-173, hess-27-2075-2023, hess-25-6133-2021}. Using machine learning and deep learning methods to construct IDF curves is a relatively new approach to achieve adaptive IDF curves, with only a handful of studies that have emerged recently exploring this alternative~\cite{idfkoya}

\vspace{1em}

Machine learning and deep learning methods have only recently started to appear in literature as an alternative to the traditional statistical methods used to construct IDF curves~\cite{idfkoya}, this trend still needs to gain traction, and more studies are needed to explore the potential of these methods. Per the universal approximation theorem (UAT), deep learning models are by design capable of capturing non-linear relationships in data, and are capable to estimate any function or distribution given enough data~\cite{Goodfellow2016-sect6.4.1}. However, artificial neural networks on their own still struggle with modeling stochastic processes and uncertainty inherent in some datasets~\cite{thacker2020fundamentalissuesregardinguncertainties}. This has led to the development of neural network architectures that are specifically designed to handle stochastic processes and uncertainty, some by design, such as Bayesian Neural Networks (BNN) and Temporal Convolutional Attention Networks (TCANs).~\cite{lin2021tcan, goan2020} 

\vspace{1em}

Machine learning techniques were initially applied to precipitation data preprocessing and temporal disaggregation tasks rather than direct IDF curve construction~\cite{geosciences9050209}. Artificial neural networks first emerged in hydrology applications during the early 1990s, primarily focusing on rainfall-runoff modeling, stream flow forecasting, and water quality analysis~\cite{doi:10.1061}. Early applications of neural networks to precipitation disaggregation began appearing in the literature around 2000, with studies exploring the use of feed-forward neural networks and competitive learning algorithms to disaggregate hourly rainfall data into sub-hourly time increments~\cite{geosciences9050209,doi:10.1061}. Using the daily rainfall data as input, and the observed subdaily resolutions as labels, the ML models would learn to estimate the downscaled precipitation data, which then could be used to construct IDF curves for different durations using the known distributions~\cite{geosciences9050209}.
The authors in this paper used machine learning in the preprocessing of the IDF generation process. Furthermore, the authors relied on data from two-thousand ground stations in the United States, and did not leverage any remotely sensed precipitation data for this task.

\vspace{1em}

One of the first comprehensive studies comparing machine learning and deep learning models with traditional statistical methods for direct IDF curve construction was published just recently~\cite{idfkoya}. The authors collected rainfall data from Koya, Iraq from 2005 to 2022, consisting of multiple temporal resolutions from 5-minute to 1440-minute intervals~\cite{idfkoya} and compared Linear Regression (LR), Support Vector Regression (SVR), and Long Short-Term Memory (LSTM) models against the traditional Gumbel distribution for constructing IDF curves~\cite{idfkoya}. The study revealed that the LSTM model significantly outperformed other approaches, achieving the lowest RMSE (1.44 mm/hr), lowest MAE (0.81 mm/hr), and highest R² (0.99) values compared to the Gumbel distribution which had an RMSE of 9.13 mm/hr~\cite{idfkoya}. This research represents a significant advancement in applying artificial intelligence tools directly to IDF curve predictions, marking the first successful application of recurrent neural networks with LSTM architecture for this specific hydrological application~\cite{idfkoya}. The authors in this study however, relied on governmental and land station data sources, and did not leverage satellites and remote sensing for any precipitation data.