The statistical distribution functions used to construct IDF curves have been the subject of extensive research, and were found to be very effective in designing infrastructure under a stationary climate assumptions. However, the elements of variability and uncertainty in precipitation patterns have become a cause of concern, especially with the increasing impact of climate change on precipitation patterns~\cite{Cheng2014}. This calls into question the readiness of the current urban infrastructure to deal with these changes, especially since the stationarity climate assumption may lead to an underestimation of extreme precipitation by as much as 60\%~\cite{Cheng2014}. This has led to a growing interest to develop IDF construction methods that no longer rely on the stationarity assumption, but instead take into consideration the changing climate and precipitation patterns using methods such as bayesian inference or other distributions entirely~\cite{Cheng2014, hess-2020-173, hess-27-2075-2023, hess-25-6133-2021}. Using machine learning and deep learning methods to construct IDF curves is a relatively new approach to achieve adaptive IDF curves, with only a handful of studies that have emerged recently exploring this alternative~\cite{idfkoya}