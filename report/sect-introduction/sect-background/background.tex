The construction of accurate IDF curves necessitates high-resolution temporal precipitation data, particularly at sub-hourly intervals. Such data is critical for the design of urban drainage systems and for robust flood risk assessments, as short-duration extreme rainfall events can precipitate unforeseen natural disasters, potentially resulting in significant loss of life and considerable economic costs~\cite{5}. In many regions, particularly within developing countries, the limited density of rain-gauge networks constrains spatial coverage and consequently compromises the accuracy of rainfall estimates~\cite{6}. This scarcity in high-resolution data poses considerable challenges for engineers, as it hinders the acquisition of sub-hourly precipitation measurements required for the reliable construction of IDF curves, often necessitating the use of suboptimal IDF relationships~\cite{6}.

\vspace{1em}

The continuous advancement in satellite precipitation retrieval makes it increasingly important to develop methods that incorporate these datasets in constructing IDF curves, especially in regions where in-situ rainfall observations are sparse and have insufficient record length as it offers a credible and often superior alternative source for sub-hourly precipitation information in data-sparse regions~\cite{7}. Satellite-based precipitation products such as the Global Precipitation Measurement (GPM) Integrated Multi-satellitE Retrievals for GPM (IMERG), the Global Satellite Mapping of Precipitation Moving Vector with Kalman Filter (GSMaP-MVK), the Precipitation Estimation from Remotely Sensed Information using Artificial Neural Networks (PERSIANN), and others have demonstrated effectiveness in estimating IDF curves, with some products showing strong correlation with ground observations~\cite{7, 8}. Half-hourly precipitation data from GPM-IMERG, for instance, was used to derive regional IDF curves with reasonable accuracy—especially when using the Gumbel distribution.~\cite{8}