\begin{abstract}
  Intensity, Duration and Frequency (IDF) curves have long used statistical distributions to estimate precipitation extremes, but with climate change showing its full impact, these methods struggle with the increasing variability of precipitation patterns. As a result, this has led to a growing interest in using machine and deep learning techniques to improve precipitation estimation.
  Often the areas impacted most by climate change are data-sparse developing regions, where ground-based meteorological data is limited. Satellite-based precipitation products, such as the Global Precipitation Measurement (GPM) Integrated Multi-satellite Retrievals (IMERG), provide a promising alternative, giving access to sub-hourly precipitation estimates globally.
  This study presents a comprehensive methodology for IDF curve development utilizing artificial intelligence techniques applied to remotely sensed precipitation data from Beirut, Lebanon. Precipitation data was collected from GPM-IMERG satellite observations over Beirut International Airport, encompassing multiple temporal resolutions. Machine learning and deep learning architectures, including Support Vector Regression (SVR), Artificial Neural Networks (ANN), and novel Temporal Convolutional Networks (TCN) enhanced with self-attention mechanisms, are evaluated and benchmarked against conventional statistical distributions, specifically the Gumbel distribution.
  Experimental results demonstrate that SVR and ANN models yield IDF curves comparable to traditional statistical approaches, while TCN models with self-attention mechanisms exhibit superior performance in capturing extreme precipitation events, providing enhanced resolution of tail behavior in the probability distribution.
\end{abstract}

\noindent\textbf{Keywords:} IDF curves, machine learning, deep learning, precipitation estimation, GPM-IMERG, temporal convolutional networks, self-attention, support vector regression, artificial neural networks, remote sensing, artificial intelligence.