\begin{abstract}
\noindent Intensity-Duration-Frequency (IDF) curves describe the relationship between rainfall intensity, the duration of rainfall events, and the frequency with which these events occur at a specific location. These curves are fundamental in urban drainage system design, flood risk analysis, and the planning of hydraulic infrastructure as they provide engineers with critical data to assess the likelihood of extreme rainfall events.
IDF curves have traditionally relied on statistical distributions to model precipitation extremes. However, conventional methodologies face significant limitations in addressing the increasing variability and uncertainty inherent in precipitation patterns, particularly during the construction phase of these curves.
This study aims to leverage satellite-based precipitation datasets and advanced machine learning techniques to develop more accurate and robust IDF curves in real-time, thereby lowering uncertainty in the curves and improving the reliability of IDF curve construction under non-stationary conditions.
For this, daily precipitation data are collected from Global Precipitation Measurement (GPM) satellite observations over Beirut, with multiple temporal resolutions.
Annual maximum rainfall values from 1998 to 2024 are derived from these data points to develop machine learning and deep learning architectures, including Support Vector Regression (SVR), Artificial Neural Networks (ANN), novel Temporal Convolutional Networks (TCN), including one enhanced with sparse self-attention mechanisms (TCAN), which learn distributions used to generate IDF curves.
Compared to the Gumbel distribution, TCAN models achieved the highest accuracy (R$^{2}$ = 0.998) in generating IDF curves, surpassing TCN (R$^{2}$ = 0.996), SVR (R$^{2}$ = 0.96), and ANN (R$^{2}$ = 0.93), and offering superior modeling of extreme precipitation and tail distribution behavior.
This research represents one of the first efforts to directly employ artificial intelligence techniques to directly construct IDF curves without statistical distributions, presenting a more adaptive approach and addressing the need for reliable and resilient IDF curves to use in urban planning.
\end{abstract}

\noindent\textbf{Keywords:} Intensity-Duration-Frequency curves, Machine Learning, Remote Sensing.