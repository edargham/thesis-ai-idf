\begin{abstract}
\noindent Intensity-Duration-Frequency (IDF) curves are tools that establish the relationship between rainfall intensity (I), duration (D), and frequency (F) of occurrence for a specific geographic location or catchment.
These curves find their most critical application in urban drainage design, flood risk assessment, and hydraulic infrastructure planning, providing engineers and hydrologists with essential data to correlate extreme rainfall intensity, duration, and recurrence intervals for climate-resilient projects.
IDF curves have traditionally relied on statistical distributions to model precipitation extremes. However, conventional methodologies face significant limitations in addressing the increasing variability and uncertainty inherent in precipitation patterns, particularly during the construction phase of these curves.
These pitfalls persist despite the widespread adoption of IDF curves as a primary design criterion for critical urban infrastructure, including drainage systems and flood mitigation structures. 
This has led to a growing interest in using machine and deep learning techniques to improve precipitation estimation accuracy and certainty.
This study presents a comprehensive methodology for IDF curve development using artificial intelligence techniques applied to remotely sensed precipitation data for Beirut, Lebanon.
Precipitation data is collected from the Global Precipitation Measurement (GPM) satellite observations over Beirut International Airport, encompassing multiple temporal resolutions.
Maximum rainfall yields for each year from 1998 to 2024 are derived from this to develop Machine learning and deep learning architectures, including Support Vector Regression (SVR), Artificial Neural Networks (ANN), and novel Temporal Convolutional Networks (TCN) enhanced with self-attention mechanisms.
The SVM and TCN models are then trained on the precipitation data to learn a distribution that will be used to generate the IDF curves.
The machine learning generated IDF curves are then compared to IDF curves generated by the Gumbel Distribution on the same data used for training.
Results demonstrate that SVR and ANN models yield more precise IDF curves compared to traditional statistical approaches, while TCN models with self-attention mechanisms exhibit superior performance in capturing extreme precipitation events, providing enhanced resolution of tail behavior in the probability distribution.
This research represents one of the pioneering efforts to directly employ novel artificial intelligence techniques to estimate extreme value probability distribution functions for IDF curve generation, presenting a more adaptive approach where traditional statistical methods struggle with the uncertain and variable nature of precipitation events and addressing the need for reliable IDF curves to use in urban planning.
\end{abstract}

\noindent\textbf{Keywords:} Intensity-Duration-Frequency (IDF) curves, rainfall intensity, rainfall duration, frequency analysis, Gumbel distribution, machine learning, deep learning, Support Vector Regression (SVR), Artificial Neural Networks (ANN), Temporal Convolutional Networks (TCN), self-attention mechanisms, remotely sensed precipitation, Global Precipitation Measurement (GPM)