\section{Literature Review}
The construction of accurate Intensity-Duration-Frequency (IDF) curves necessitates high-resolution temporal precipitation data, particularly at sub-hourly intervals. Such data is critical for the design of urban drainage systems and for robust flood risk assessments, as short-duration extreme rainfall events can precipitate unforeseen natural disasters, potentially resulting in significant loss of life and considerable economic costs~\cite{hess-28-375-2024}.
In many regions, particularly within developing countries, the limited density of rain-gauge networks constrains spatial coverage and consequently compromises the accuracy of rainfall estimates~\cite{basumatary2016}. This scarcity in high-resolution data poses considerable challenges for engineers, as it hinders the acquisition of sub-hourly precipitation measurements required for the reliable construction of IDF curves, often necessitating the use of suboptimal IDF relationships~\cite{basumatary2016}.

The continuous advancement in satellite precipitation retrieval makes it increasingly important to develop methods that incorporate these datasets in constructing IDF curves, especially in regions where in-situ rainfall observations are sparse and have insufficient record length as it offers a credible and often superior alternative source for sub-hourly precipitation information in data-sparse regions~\cite{ombadi2018}. Satellite-based precipitation products such as the Global Precipitation Measurement (GPM) Integrated Multi-satellitE Retrievals for GPM (IMERG), the Global Satellite Mapping of Precipitation Moving Vector with Kalman Filter (GSMaP-MVK), the Precipitation Estimation from Remotely Sensed Information using Artificial Neural Networks (PERSIANN), and others have demonstrated effectiveness in estimating IDF curves, with some products showing strong correlation with ground observations~\cite{ombadi2018, rs14195032}. Half-hourly precipitation data from GPM-IMERG, for instance, be used to derive regional intensity–duration–frequency (IDF) curves with reasonable accuracy—especially when using the Gumbel distribution.~\cite{rs14195032}

\vspace{1em}

The statistical distribution functions used to construct IDF curves have been the subject of extensive research, and were found to be very effective in designing infrastructure under a stationary climate assumptions. However, the elements of variability and uncertainty in precipitation patterns have become a cause of concern, especially with the increasing impact of climate change on precipitation patterns~\cite{Cheng2014}. This calls into question the readiness the current urban infrastructure to deal with these changes, especially since the stationarity climate assumption may lead to an underestimation of extreme precipitation by as much as 60\%~\cite{Cheng2014}. This has led to a growing interest to develop IDF construction methods that no longer rely on the stationarity assumption, but instead take into consideration the changing climate and precipitation patterns using methods such as bayesian inference or other distributions entirely~\cite{Cheng2014, hess-2020-173, hess-27-2075-2023, hess-25-6133-2021}. Using machine learning and deep learning methods to construct IDF curves is a relatively new approach to acheive adaptive IDF curves, with only a handful of studies that have emerged recently exploring this alternative~\cite{idfkoya}

\vspace{1em}

Machine Learning and Deep Learning methods have only recently started to appear in literature as an alternative to the traditional statistical methods used to construct IDF curves~\cite{idfkoya}, this trend still needs to gain traction, and more studies are needed to explore the potential of these methods. Per the universal approximation theorem (UAT), deep learning models are by design capable of capturing non-linear relationships in data, and are capable to estimate any function or distribution given enough data~\cite{Goodfellow2016-sect6.4.1}. However, artificial neural networks on their own still struggle with modeling stochastic processes and uncertainty inherent in some datasets~\cite{thacker2020fundamentalissuesregardinguncertainties}. This has lead to the development of neural network architectures that are specifically designed to handle stochastic processes and uncertainty, some by design, such as Bayesian Neural Networks (BNN) and Temporal Convolutional Attention Networks (TCANs).~\cite{lin2021tcan, goan2020} 

\vspace{1em}

Machine Learning techniques were initially used to preprocess precipitation data\dots Hu and Ayyub (2019) introduced a machine learning-based approach for temporally downscaling daily rainfall observations, aiming to improve the generation of intensity-duration-frequency (IDF) curves under changing climate conditions~\cite{geosciences9050209}. Utilizing a comprehensive dataset comprising daily rainfall records from approximately two thousand stations across the United States, their study addresses the prevailing focus on spatial downscaling in precipitation research by instead emphasizing temporal downscaling methods~\cite{geosciences9050209}.

The primary input to the machine learning model is the daily rainfall data, as the authors stated that many global climate models (GCMs) and regional climate models (RCMs) offer only daily rainfall data rather than sub-daily precipitation data. This apporach makes the method widely applicable to leverage readily available daily climate projections to generate IDF curves.~\cite{geosciences9050209} using the daily rainfall data as input, and the observed subdaily resolutions as labels, the ML models would learn to estimate the downscaled precipitation data, which then could be used to construct IDF curves for different durations using the known distributions.

The authors in this paper used machine learning in the preprocessing of the IDF generation process. The data is not being used to replace any of the statistical distribution used to generate IDF curves from  annual maximas and extreme rainfall events, but rather in the steps before that to prepare the data. Furthermore, the authors relied on data from two-thousand ground stations in the United States, and did not leverage any remotely sensed precipitation data for this task.

\vspace{1em}

In the paper ``Utilizing Machine Learning and Deep Learning for Precise Intensity-Duration-Frequency (IDF) Curve Predictions''~\cite{idfkoya} Ameen, Sheeraz M. \emph{et.~al.} compared several machine learning (ML) and deep learning (DL) models with traditional statistical models using the Gumbel distribution to construct IDF curves.

The initial intuition is that Gumbel only performs well on stationary data that doesn't change over time, but fails to capture the changes in rainfall trends due to climate change, something that ML and DL are well suited for.~\cite{idfkoya}

The authors collected rainfall data in Koya, Iraq from 2005 to 2022, from multiple governmental and institutional sources, consisting of 5-minute, 10-minute, 20-minute, 30-minute, 60-minute, 120-minute, 180-minute, 360-minute, 720-minute and 1440-minute intervals. Using this data, the authors from one end, used Gumbel to construct the IDF curves to establish a benchmark, and from the other end, train Linear Regression (LR), Support Vector Regression (SVR) and Long Short-Term Memory (LSTM) models to predict the IDF curves.~\cite{idfkoya}

To train the models, the authors split the rainfall data into a training set, using the data from 2005 to 2015 and a validation set from 2016 to 2022. The metrics used to evaluate each model were the Mean Absolute Error (MAE), Root Mean Square Error (RMSE) and R-squared (R$^{2}$) values. The study revealed that the LSTM model outperformed the other ML and DL model as well as the Gumbel distribution, achieving the lowest RMSE (1.44 mm/hr), lowest MAE (0.81 mm/hr) and the Highest R$^{2}$ (0.99) values.~\cite{idfkoya}

The authors in this study however, relied on governmental and station data sources, and did not source any data using remote sensing technologies or satellite data.