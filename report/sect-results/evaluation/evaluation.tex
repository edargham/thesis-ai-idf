\subsection{Model Evaluation}
The performance of the machine learning and deep learning models for IDF curve construction is quantitatively assessed using several commonly adopted metrics: RMSE, MAE, R$^{2}$, and NSE\@. These metrics are computed on the test set and are summarized in Table~\ref{tab:performance}.
The IDF curves generated by the different models are illustrated in Figure~\ref{fig:combined_idf_curves}, showing the rainfall intensity as a function of event duration for various return periods.

\vspace{1em}

SVM produces an R$^{2}$ of 0.903, indicating reasonable predictive accuracy for rainfall intensity as a function of event duration, but underperforming relative to neural and convolutional architectures. ANN improves the overall fit to the observed data, yielding an R$^{2}$ of 0.939 and reduced error values. TCN achieves substantially superior results, lowering both RMSE and MAE while reaching an R$^{2}$ of 0.966, attributable to enhanced modeling of temporal dependencies in rainfall records. TCAN also demonstrates strong predictive capability (R$^{2}$ = 0.962), with minor trade-offs in error metrics compared to plain TCN\@. TCAN, however, uses significantly less hidden channels to achieve this.
\vspace{1em}